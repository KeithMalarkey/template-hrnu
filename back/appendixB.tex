% !TEX root = ../main.tex

% 附录B
\chapter{这个星球上最好的免费Linux软件列表}[List of the Best Linux Software in our Planet]
\section{系统}

\href{http://fvwm.org/}{FVWM 自从上世纪诞生以来,此星球最强大的窗口管理器。}
推荐基于FVWM的桌面设计hifvwm:\href{https://github.com/dustincys/hifvwm}{https://github.com/dustincys/hifvwm}。

\subsection{hifvwm的优点}

\begin{enumerate}
	\item 即使打开上百个窗口也不会“蒙圈”。计算机性能越来越强大,窗口任务的管理必须要升级到打怪兽级别。
	\item 自动同步Bing搜索主页的壁纸。每次电脑开机,午夜零点自动更新,用户
	      也可以手动更新,从此审美再也不疲劳。
	\item 切换窗口自动聚焦到最上面的窗口。使用键盘快捷键切换窗口时候,减少
	      操作过程,自动聚焦到目标窗口。这一特性是虚拟窗口必须的人性化设
	      计。
	\item 类似window右下角的功能的最小化窗口来显示桌面的功能此处类似
	      win7/win10,实现在一个桌面之内操作多个任务。
	\item 任务栏结合标题栏。采用任务栏和标题栏结合,节省空间。
	\item 同类窗口切换。可以在同类窗口之内类似alt-tab的方式切换。
	\item ……
\end{enumerate}

\section{其他}

\href{https://orgmode.org/}{orgmode,最强大的笔记系统,从来没有之一。}

\href{https://www.jianguoyun.com/}{坚果云,国内一款支持WebDav的云盘系统,国内真正的云盘没有之一。}

\section{vim}
% 实现中英文每一句一行,以及实现每一句折叠断行的简单正则式,tex源码更加乖乖。
不用在乎,这只是一段\TeX 语句,论文中用不到,注意语言设定,你可以在.syl文件中设定不同code语言的风格。
\begin{lstlisting}[language = TeX]
vnoremap <leader>fae J:s/[.!?]\zs\s\+/\="\r".matchstr(getline('.'), '^\s*')/g<CR>
vnoremap <leader>fac J:s/[。!?]/\=submatch(0)."\n".matchstr(getline('.'), '^\s*')/g<CR>
vnoremap <leader>fle :!fmt -80 -s<CR>
\end{lstlisting}
