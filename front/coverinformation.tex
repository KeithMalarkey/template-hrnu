% !TEX root = ../main.tex

% 专硕请取消下面注释
% \makeatletter
%   \def\hitsz@csubjecttitle{类别}
% \makeatother

\hitszsetup{
	%******************************
	% 注意:
	%   1. 配置里面不要出现空行
	%   2. 不需要的配置信息可以删除
	%******************************
	%
	%=====
	% 秘级
	%=====
	sid={2022300787},
	natclassifiedindex={TP391},
	% intclassifiedindex={62-5},
	%
	%=========
	% 中文信息
	%=========
	% ctitleone={局部多孔质气体静压轴承},%本科生封面使用
	% ctitletwo={关键技术的研究},%本科生封面使用
	ctitlecover={基于多尺度特征联合表示的\\高光谱图像分类方法研究},%! 不超过25字,放在封面中使用,自由断行
	ctitle={基于多尺度特征联合表示的高光谱图像分类方法研究},%放在原创性声明中使用
	% csubtitle={一条副标题}, %一般情况没有,可以注释掉
	cxueke={工学},
	cpostgraduatetype={学术}, % 学术/专业
	csubject={计算机科学与技术},
	csubjectdir={计算机应用技术},
	% csubject={机械工程},
	caffil={计算机科学与信息工程学院},
	% caffil={哈尔滨工业大学(深圳)},
	cauthor={温晓彦},
	csupervisor={于晓冬 教授},
	% cassosupervisor={某某某 教授}, % 副导师
	% ccosupervisor={某某某 教授}, % 合作导师
	% 日期自动使用当前时间,若需指定按如下方式修改:
	cdate={2025年5月},
	% 指定第二页封面的日期,即答辩日期
	cdatesecond={2025年05月09日},
	% cstudentid={SZ160310217},
	% cstudenttype={同等学力人员}, %非全日制教育申请学位者
	%(同等学力人员)、(工程硕士)、(工商管理硕士)、
	%(高级管理人员工商管理硕士)、(公共管理硕士)、(中职教师)、(高校教师)等
	%
	%
	%=========
	% 英文信息
	%=========
	etitle={Hyperspectral Image Classification with Multi-Scale Feature Joint Representation},
	% esubtitle={This is the sub title},
	exueke={Engineering},
	esubject={Computer Science and Technology},
	eaffil={School of Computer Science and Information Engineering},
	eauthor={Xiaoyan Wen},
	esupervisor={Prof. Yu Xiaodong},
	% eassosupervisor={XXX},
	% ecosupervisor={Prof. XXX}, % Co-Supervisor off Campus
	% 日期自动生成,若需指定按如下方式修改:
	edate={May, 2025},
	estudenttype={Master of Engineering},
	%
	% 关键词用“英文逗号”分割
	ckeywords={高光谱图像分类, 多尺度语义, 联合表示, 深度学习},
	ekeywords={Hyperspectral image classification, multi-scale semantic, joint representation, deep learning},
}

% 中文摘要
% ! 500-1000字
\begin{cabstract}
	% ?话题引入
	% ?研究的必要性和重要性
	% ?技术背景对应要解决的问题
	% ?阐述论文研究的价值和意义
	% ?针对不同问题,说明意义
	随着遥感技术的不断发展和深度学习技术的深入研究,多源、多粒度的高光谱遥感图像不断涌现,传统单一尺度场景下的高光谱遥感图像分析技术已经无法胜任不同语义尺度场景下的
	分类任务,因此,高光谱遥感图像的多尺度语义学习已经成为了国内外学术界的研究重点。鉴于不同空间或光谱分辨率的高光谱图像反映了地物不同尺度的特征,因而构建并联合不同
	尺度的光谱和空间特征,是实现多尺度分类预测的必要前提。实际上,高光谱图像因其特殊的光谱-空间结构,从而为现有的深度学习方法提供了足量的数据支持。但是,采集和标注高
	光谱遥感图像数据将带来极大的经济开销,并需要专业知识的支撑。更糟糕的是,


\end{cabstract}

% 英文摘要
% ! 对应中文摘要
\begin{eabstract}
	With the continuous development of remote sensing technology and in-depth research on deep learning technology, multi-source and multi-granularity
	hyperspectral remote sensing images are constantly emerging. Conventional hyperspectral remote sensing image analysis technology in single-scale 
	scenarios can no longer meet the classification tasks in different semantic scale scenarios. Therefore, multi-scale semantic learning of hyperspectral 
	remote sensing images has become a research focus in the academic community both domestically and internationally. Given that hyperspectral images 
	with different spatial or spectral resolutions reflect the characteristics of ground objects at different scales, constructing and combining spectral 
	and spatial features at different scales is a necessary prerequisite for achieving multi-scale classification prediction. In fact, due to their 
	unique spectral-spatial structure, hyperspectral images provide ample data support for existing deep learning methods. However, collecting and
	annotating hyperspectral remote sensing image data incurs significant economic costs and requires professional knowledge.
\end{eabstract}
